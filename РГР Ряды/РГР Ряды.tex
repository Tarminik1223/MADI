\documentclass[a4paper,12pt]{article}
\usepackage[left=2cm,right=2cm,top=2cm,bottom=2cm,bindingoffset=0cm]{geometry}

%%% Работа с русским языком
\usepackage{cmap}					% поиск в PDF
\usepackage{mathtext} 				% русские буквы в фомулах
\usepackage[T2A]{fontenc}			% кодировка
\usepackage[utf8]{inputenc}			% кодировка исходного текста
\usepackage[english,russian]{babel}	% локализация и переносы
\usepackage{csquotes}               % ещё одна штука для цитат
\usepackage{amsmath}

%% Шрифты
\usepackage{euscript}	            % Шрифт Евклид
\usepackage{mathrsfs}               % Красивый матшрифт
\usepackage{amsfonts}

%%% Работа с картинками
\usepackage{graphicx}               % Для вставки рисунков
\setlength\fboxsep{3pt}             % Отступ рамки \fbox{} от рисунка
\setlength\fboxrule{1pt}            % Толщина линий рамки \fbox{}
\usepackage{wrapfig}                % Обтекание рисунков и таблиц текстом


\begin{document}
\begin{flushleft}
	\Large{РГР Ряды\\ Тарасов М. группа 2бПМ2\\ Вариант 10}
\end{flushleft}
	
\section*{№1}
	$\sum\limits_{n=1}^{\infty} \frac{1}{3n+5}$; $a_n = \frac{1}{3n+5}$\\
	Рассмотрим ряд $\sum\limits_{n=1}^{\infty} \frac{1}{n}$ ($b_n = \frac{1}{n}$) - расходящийся:\\
	$\lim\limits_{n\to\infty} \frac{b_n}{a_n} = \lim\limits_{n\to\infty} \frac{3n+5}{n} = 3 \in \mathbb{R} \setminus \{0\}$\\
	$\Rightarrow$ ряды $\sum\limits_{n=1}^{\infty} \frac{1}{3n+5}$ и $\sum\limits_{n=1}^{\infty} \frac{1}{n}$ ведут себя одинакого\\
	$\Rightarrow$ по предельному признаку сравнения ряд $\sum\limits_{n=1}^{\infty} \frac{1}{3n+5}$ расходистся
	
\section*{№2}
	$\sum\limits_{n=1}^{\infty} \frac{1}{\sqrt{n^{10}+n^2}}$; 
	$a_n = \frac{1}{\sqrt{n^{10}+n^2}}$\\
	Рассмотрим ряд 
	$\sum\limits_{n=1}^{\infty} \frac{1}{n^5}$ ($b_n=\frac{1}{n^5}$)
	- сходящийся (ряд Дирихле, 5>1):\\
	$
	\lim\limits_{n\to\infty} \frac{b_n}{a_n} = 
	\lim\limits_{n\to\infty} \frac{\sqrt{n^{10}+n^2}}{n^5} = 
	\lim\limits_{n\to\infty} \frac{n^5\sqrt{1+\frac{1}{n^8}}}{n^5} = 
	\lim\limits_{n\to\infty} \sqrt{1+\frac{1}{n^8}} = 1\in \mathbb{R} \setminus \{0\}$\\
	$\Rightarrow$ ряды ведут себя одинакого\\
	$\Rightarrow$ по предельному признаку сравнения ряд $\sum\limits_{n=1}^{\infty} \frac{1}{\sqrt{n^{10}+n^2}}$ сходится
	
\section*{№3}
	$\sum\limits_{n=1}^{\infty} \frac{1}{2^n+n}$; 
	$a_n = \frac{1}{2^n+n}$\\
	Рассмотрим ряд $\sum\limits_{n=1}^{\infty} \frac{1}{2^n}$;
	$b_n = \frac{1}{2^n}$:\\
	$
	\lim\limits_{n\to\infty} \frac{b_{n+1}}{b_n} =
	\lim\limits_{n\to\infty} \frac{2^n}{2\cdot2^n} = \frac{1}{2} < 1
	$\\
	$\Rightarrow$ по признаку Даламбера ряд $\sum\limits_{n=1}^{\infty} \frac{1}{2^n}$ сходится\\	
	$\forall n  \hookrightarrow \frac{1}{2^n+n} < \frac{1}{2^n}$
	и ряд $\sum\limits_{n=1}^{\infty} \frac{1}{2^n}$ сходится\\
	$\Rightarrow$ по признаку сравнения ряд $\sum\limits_{n=1}^{\infty} \frac{1}{2^n+n}$
	сходится

\section*{№4}
	$\sum\limits_{n=0}^{\infty} \frac{1}{(2n+1)!}$;
	$a_n = \frac{1}{(2n+1)!}$\\
	$
	\lim\limits_{n\to\infty} \frac{a_{n+1}}{a_n} =
	\lim\limits_{n\to\infty} \frac{(2n+1)!}{(2n+3)!}=
	\lim\limits_{n\to\infty} \frac{(2n+1)!}{(2n+1)!(2n+1)(2n+3)} =
	\lim\limits_{n\to\infty} \frac{1}{(2n+2)(2n+2)} = 0 < 1
	$\\
	$\Rightarrow$ по признаку Даламбера ряд 
	$\sum\limits_{n=0}^{\infty} \frac{1}{(2n+1)!}$ сходится
	
\section*{№5}
	$\sum\limits_{n=1}^{\infty} \frac{(-1)^n}{\sqrt{2n+1}}$;
	$a_n = \frac{1}{\sqrt{2n+1}}$\\
	Проверим ряд на абсолютную сходимость:\\
	$\forall n \in\mathbb{N} \hookrightarrow \frac{1}{\sqrt{2n+1}} > \frac{1}{2n+1}$\\
	$\sum\limits_{n=1}^{\infty} \frac{1}{2n+1} \sim \sum\limits_{n=1}^{\infty} \frac{1}{n}$ (Гармонический)\\
	$\Rightarrow$ ряд $\sum\limits_{n=1}^{\infty} \frac{1}{2n+1}$ расходится по предельному признаку сравнения\\
	$\Rightarrow$ по признаку сравнения ряд $\sum\limits_{n=1}^{\infty} \frac{1}{\sqrt{2n+1}}$ расходится\\
	Исследуем ряд на условную сходимость:\\
	1) $\frac{1}{\sqrt{2n+1}} \searrow$\\
	2) $\lim\limits_{n\to\infty} a_n = 
	\lim\limits_{n\to\infty} \frac{1}{\sqrt{2n+1}} =
	\lim\limits_{n\to\infty} \frac{1/\sqrt{n}}{\sqrt{2+1/n}} = 0$\\
	$\Rightarrow$ по признаку Лейбница ряд 
	$\sum\limits_{n=1}^{\infty} \frac{(-1)^n}{\sqrt{2n+1}}$ сходится условно
	
\section*{№6}
	$\sum\limits_{n=1}^{\infty} \frac{(-1)^nn!}{e^n}$;
	$a_n = \frac{n!}{e^n}$\\
	Проверим ряд на абсолютную сходимость:\\
	$\lim\limits_{n\to\infty} \frac{a_{n+1}}{a_n} =
	\lim\limits_{n\to\infty} \frac{(n+1)!\cdot e^n}{e^{n+1}\cdot n!} =
	\lim\limits_{n\to\infty} \frac{n!(n+1)\cdot e^n}{e^n\cdot e\cdot n!} = 
	\lim\limits_{n\to\infty} \frac{n+1}{e} = \infty
	$\\
	$\Rightarrow$ по признаку Даламбера ряд $\sum\limits_{n=1}^{\infty} \frac{n!}{e^n}$ расходится\\
	Исследуем ряд на условную сходимость:\\
	1) $\frac{n!}{e^n} \nearrow$\\
	$\Rightarrow$ по признаку Лейбница ряд 
	$\sum\limits_{n=1}^{\infty} \frac{(-1)^nn!}{e^n}$ расходится
	
\section*{№7}
	$\sum\limits_{n=1}^{\infty} \frac{(-1)^n}{n^2+n}$;
	$a_n = \frac{1}{n^2+n}$\\
	Проверим ряд на абсолютную сходимость:\\
	Рассмотрим ряд $\sum\limits_{n=1}^{\infty} \frac{1}{n^2} (b_n = \frac{1}{n^2})$
	- сходящийся (ряд Дирихле, 2>1)\\
	$\forall n \in \mathbb{N} \hookrightarrow \frac{1}{n^2+n} < \frac{1}{n^2}$ 
	и ряд $\sum\limits_{n=1}^{\infty} \frac{1}{n^2}$ сходится\\
	$\Rightarrow$ по признаку сравнения ряд 
	$\sum\limits_{n=1}^{\infty} \frac{1}{n^2+n}$ сходится\\
	$\Rightarrow$ ряд $\sum\limits_{n=1}^{\infty} \frac{(-1)^n}{n^2+n}$ сходится абсолютно
	
\newpage
\section*{№8}
	$\sum\limits_{n=2}^{\infty} \frac{(-1)^n}{nln^3n}$;
	$a_n = \frac{1}{nln^3n}$\\
	Проверим ряд на абсолютную сходимость:\\
	$
	\int\limits_2^\infty \frac{1}{xln^3x}dx = 
	\int\limits_2^\infty \frac{d(lnx)}{ln^3x} =
	\lim\limits_{b\to\infty} \left. {\frac{ln^{-2}x}{-2}} \right|_2^b =
	-\frac{1}{2}\lim\limits_{b\to\infty} \left. {\frac{1}{ln^2x}} \right|_2^b =
	-\frac{1}{2}\lim\limits_{b\to\infty} (\frac{1}{ln^2b}-\frac{1}{ln^22}) = 
	\frac{1}{2}\cdot \frac{1}{ln^22} \in \mathbb{R}\\
	$
	$\Rightarrow$ по интегральному признаку Коши ряд $\sum\limits_{n=2}^{\infty} \frac{(-1)^n}{nln^3n}$ сходится абсолютно
	
	
\end{document}